\documentclass[10pt,reqno]{amsart}
\usepackage{articlesetup}
\usepackage{float}

\bibliography{bibliography}

\author{Harvey J. Stein}
\email{hjstein@bloomberg.net}
\address{Bloomberg L.P., New York, NY, USA}

\date{\today}

\begin{document}

\title{COVID-19: Garbage In, Garbage Out}

\begin{abstract}
  COVID-19 data discussion.
\end{abstract}

\maketitle
\tableofcontents

\section{Introduction}
\label{sec:intro}
If you recall my 4/21 analysis of the NYC COVID-19 data
(\href{https://hjstein.blogspot.com/2020/04/covid-19-nyc-stats-not-what-they-seem.html}{COVID-19 NYC Stats -- Not What They Seem}),
you'd remember a graph showing the extent to which missing
data has a major impact on the reported incident counts.
Figure~\ref{fig:daily} has an updated version.

\begin{figure}[H]
  \centering
  \includegraphics[width=\textwidth]{../Notebooks/casesPerDayHistoryRaw.png}
  \caption{NYC COVID-19 cases per day from each daily report.  Data
    from the NYC COVID-19 data github repository.}
  \label{fig:daily}
\end{figure}

As you can see, the peak hasn't moved from April 7th, but we're still
getting data for dates as far back as March!
Figure~\ref{fig:smoothDaily} has the updated 7 day average report,
which shows more clearly that three days ago additional incidents were
reported for every date going back as far as March 25th.

\begin{figure}[H]
  \centering
  \includegraphics[width=\textwidth]{../Notebooks/casesPerDayHistory.png}
  \caption{7-day rolling average of NYC COVID-19 cases per day from each daily report.  Data
    from the NYC COVID-19 data github repository.}
  \label{fig:smoothDaily}
\end{figure}

After seeing this analysis, Jon Asmundsson, editor of {\it Bloomberg
  Markets}, wrote back to me:
\begin{quotation}
  Your analysis is really interesting.  Have you looked at other states/cities?
\end{quotation}

This led me to try.

\section{Garbage in?}
My first stop was the
\href{https://github.com/owid/covid-19-data}{Our World in Data github
  repository}.  I
\href{https://github.com/hjstein/covid-19-data}{forked the
  repository}, imported my analysis code, extracted the historical
reports and graphed the history.  The results are in figure~\ref{fig:owid}

\begin{figure}[H]
  \centering
  \includegraphics[width=\textwidth]{../../covid-19-data/scripts/notebooks/casesPerDayHistory.png}
  \caption{USA COVID-19 cases per day from each daily report.  Data
    from the OWID COVID-19 data github repository.}
  \label{fig:owid}
\end{figure}

It's nice that the 7 day average is dropping, but where are the new
reports?  There are no updates -- no missing data!  How could it be
that the data the USA cases counts for yesterday that they receive
today are complete?  Given what we know about the NYC data, and given
that the NYC data is part of the USA data, it can't possibly be the
case that on a given day they know exactly the number of cases the day
before.  Something odd must be going on.

So I entered an
\href{https://github.com/owid/covid-19-data/issues/41}{issue on the
  OWID COVID-19 github repository}.  I asked how they generate the
data.  Edouard Mathieu, Data Manager at OWID, responded:
\begin{quotation}
  For confirmed cases and deaths, our data comes from the European
  Centre for Disease Prevention and Control (ECDC). We discuss how and
  when the ECDC collects and publishes this data \href{https://ourworldindata.org/coronavirus#our-world-in-data-relies-on-data-from-the-european-cdc}{here}.

  Importantly, the ECDC follows a general rule of not changing past
  values in its data. If cases/deaths are reported with a lag—a general
  lag, as you described, or occasional
  \href{https://www.theguardian.com/us-news/2020/apr/15/new-york-city-coronavirus-death-toll-jumps-revised-count}{'blocks'
    of new data}--these new
  cases/deaths will be added \bf{on the date that the country reported them
    to the ECDC.}
\end{quotation}

So, OWID gets their data from the ECDC -- The European Center for
Disease Prevention and Control, and the ECDC doesn't collect data by
incident date, it collects the data by the date on which it receives
the reports.

Further research showed that it's not just the ECDC.  The
\href{https://github.com/CSSEGISandData/COVID-19}{Johns Hopkins
  University COVID-19 repository}, and the
\href{https://github.com/nytimes/covid-19-data}{New York Times
  COVID-19 repository} also record instances by report receipt date
instead of by incident date.  And these are the major sources of data
that people use for modeling, for planning disease responses and for
reporting.

I followed up with Edouard Mathieu.  I asked him if he knew of any
rationale for why the data was being collected this way.  He wrote:

\begin{quotation}
  Hi Harvey,

Thanks for getting in touch. There's no perfectly clear and obvious
answer but here's how I understand it: it comes down to who is
publishing that data and what they consider their "role" to be.

- If they're a government, ministry, public health authority,
etc. whose job it is to report on the situation, then they try to give
the most accurate picture of how the epidemic evolved in the country,
including by going "back in time" and correcting past figures to make
them as close as possible to reality.

- On the other hand, organizations like the WHO, ECDC, JHU, consider
themselves to be data providers first and foremost. This means they
aim for "stability" in their data, and they avoid as much as possible
(or even completely) going back and changing past days, as the many
people/applications/dashboards/etc. relying on their data wouldn't
necessarily notice these changes and handle them correctly. 

Another issue is that while it's easy for 1 country to fix past
figures in a time series on its own website, it would be much harder
for an international organization that receives data from 200+
countries to accept retroactive changes—their job is made a lot easier
by simply telling countries to send data corrections as if they were
big "blocks" of cases or deaths that suddenly afppear on a given day.

This is obviously an issue sometimes, especially when some of those
corrections are very large (for example New York City or China in
April), but we don't know of any large and reliable data provider that
reports in the way you're looking for.

Best,

Edouard
\end{quotation}

I also contacted Lauren Gardner, Associate Professor, Department of
Civil and Systems Engineering, Co-Director of Center for Systems
Science and Engineering (CSSE), Johns Hopkins University.  Professor
Gardner and her team are responsible for the
\href{https://coronavirus.jhu.edu/map.html}{Johns Hopkins COVID-19
  Dashboard}.  She wrote:

\begin{quotation}
  And yes, there are absolutely issues with using reporting data
  rather than incidence rate, however more often than not, that is all
  that is available.
\end{quotation}

\section{Garbage out?}

So what's the big deal?  Counting by report date instead of incident
essentially takes some percentage of the actual data and moves it
later in time.  This flattens the curve.  As a result, it makes the
infection rate appear lower and it makes the peak appear later.
Moreover, since sites will report a number of days together, it also
makes the data jumpier and thus harder to analyze.

The problem is that scientists are using these numbers to model the
disease, the government is using these numbers to plan
how to address the risks, and the media is reporting about the
numbers.  So it reduces the accuracy of the models, interferes with
planning and leads to hysterical media reports about irrelevant rising
and falling of death counts.

So how big is the effect, really?  I calculated it by taking the NYC
data and backing out what it would look like if it was recorded by
report date instead of incident date.

\printbibliography

\end{document}
