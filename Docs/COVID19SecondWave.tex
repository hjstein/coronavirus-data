\documentclass[10pt,reqno]{amsart}
\usepackage{articlesetup}
\usepackage{float}

\bibliography{bibliography}

\author{Harvey J. Stein}
\email{hjstein@bloomberg.net}

\date{\today}

\begin{document}

\title{COVID-19 Second Wave in NYC?  Yes and No}

\begin{abstract}
  Recent data shows a jump in COVID-19 infections in NYC.  How worried
  should we be?
\end{abstract}



\maketitle
\tableofcontents

\section{Introduction}
\label{sec:intro}
Since September, NYC COVID-19 infection rates have been mostly
increasing.  This has lead to a spike in news articles, such as ``New
York City sees 'very worrisome' spike in coronavirus infection rate''
\cite{Politico2020Worrisome}.  How worrisome is it, really?  Let's
look at the data.

\section{Recent history}
\label{sec:recent}

Figure~\ref{fig:dailyRecent} shows the 7-day rolling average of the
number of cases per day since the drop off of the first wave in June.
After the drop off (due to the impact of mitigating measures, such as
the lock-down and social distancing), the number of cases per day
dropped to around 325/day.  There was a small increase in the number
of cases per day in mid-July, peaking around 375/day, after which the
number of cases per day dropped down to under 250/day.

\begin{figure}[h!tbp]
  \centering
  \includegraphics[width=\textwidth]{../Notebooks/recentHistoryNov2.png}
  \caption{NYC COVID-19 cases per day, more recent report, beginning
    of November.}
  \label{fig:dailyRecent}
\end{figure}

Starting in September, we see a rise in the number of cases per
day.  It has steadily increased, peaking at 550 cases per day on
October 5th.  Subsequently, it dropped down to 450 cases per day and started
rising again, hitting 550 cases per day again on October 29th.  Given
that this last peak was less than a week ago, we can expect it to rise
higher.

People are calling this a worrisome second wave, as exemplified in the
above article.  But, to put this rise in perspective, we need to the
data leading up to the latest peak (Figure~\ref{fig:secondPeak}) to
the data leading up to the first peak (~Figure~\ref{fig:firstPeak}).

\begin{figure}[h!tbp]
  \centering
  \includegraphics[width=\textwidth]{../Notebooks/twoMonthsToOctPeak.png}
  \caption{NYC COVID-19 cases per day, two months up to October peak.}
  \label{fig:secondPeak}
\end{figure}

\begin{figure}[h!btp]
  \centering
  \includegraphics[width=\textwidth]{../Notebooks/historyTwoMonthsToPeak.png}
  \caption{Data leading up to peak in first wave.}
  \label{fig:firstPeak}
\end{figure}

As you can see, for the recent peak, it's taken about 4 weeks to
double.  We were at 250 cases/day at the beginning of September and
didn't hit 550 cases/day until the end of the first week of October.

Compare this to the beginning of the pandemic.  On March 10th, there
were almost no cases per day.  Over the course of one week, the number
of cases per day jumped up to 1,000.  Two days later we were
experiencing 2,000 cases per day.  Three days later we hit 3,000
cases per day, and 5 days later were seeing 4,000 cases/day.  The peak
was over 5,000 cases per day.

So, compared to the first wave, the recent peak is more of a ripple
than a wave.  While we still have to be careful, it's not unexpected
that between school openings, restaurant openings and holidays, we'd
see some additional growth.  But it looks like the social distancing
measures are continuing to do their work.

To make sure, let's look at the latest data as compared to
earlier data (Figure~\ref{fig:growthNov2}).  It looks like we've
pretty much received all of the data through October 12th.  For more
recent counts, we have to wait for additional data to be received,
which would likely increase the post-October 12th counts.  In
particular, I expect the counts for the last week of October to rise
to 600 or more cases per day before dropping off.

I suspect that this recent rise is the impact of people
preparing for and celebrating Halloween.  I would expect that the
counts will subsequently drop in November and rise again around
Thanksgiving.

\begin{figure}[h!btp]
  \centering
  \includegraphics[width=\textwidth]{../Notebooks/casesPerDayHistoryNov2.png}
  \caption{Recent history, all reports.}
  \label{fig:growthNov2}
\end{figure}

\section{Summary}
The social distancing steps that have been taken appear to be
effective in keeping the cases per day reasonably low.  We should
expect the number of cases per day to stabilize at new levels as
social distancing measures are lessened, such as with the opening of
schools and restaurants.  And we should expect that rates will rise
as we approach commonly celebrated holidays and subsequently fall off.

So, it looks like the recent rise of infections per day in NYC
is to be expected, and not as worrisome as is being portrayed in the
media.  In NYC, it looks like it's more a ripple than a second wave.

\printbibliography

\end{document}
